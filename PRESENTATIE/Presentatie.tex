\documentclass{beamer}
\usepackage[utf8]{inputenc}
\usepackage{blindtext}
\usefonttheme{structurebold}
\setbeamerfont{title}{family=\rm}
\usetheme{Rochester}
\useoutertheme{smoothbars}
\blindmathtrue

\title[Something]{Hoe maak je een presentatie met latex}
\subtitle{En hoe draag je je schaamte met trots?}
\institute{Hogeschool Rotterdam}
\author{Quinten Stekelenburg}
\date{\today}

\AtBeginSection
{
  \begin{frame}
    \frametitle{Table of Contents}
    \tableofcontents[currentsection]
  \end{frame}
}

\begin{document}

\frame{\titlepage}


\frame
{
	\tableofcontents
}

\section[Introductie]{Introductie}


\frame
{
Op de vorige slides, zag je een inhoudsopgave, en een herinnering aan waar we nu zijn in de presentatie. Verder was er een hele mooie titelpagina, helemaal mooi gemaakt door latex. Aan het begin van elke section word de inhoudsopgave opnieuw weergeven, zodat je niet vergeet waar je bent. En je kan door het thema ook op de navigation bar zien in welke section je bent.
\transdissolve
}

\subsection[WOA SUBSECTIE WAT]{WOA SUBSECTIE WAT}

\frame
{
\blindtext
}

\subsubsection[WOA EEN SUBSUBSECTIE]{WAT GEBEURT ERRR}

\frame
{
\blindtext
}

\section[First Part]{First Part}

\frame
{
	\frametitle{Zo maak je een title}
	\blindtext
}

\section[Second part]{Second part}

\frame
{
\frametitle{Dit is de 2e slide}
\framesubtitle{En dit is een subtitle}
Dit is de 2e slide.
}

\frame
{
\frametitle{Afwezige tekst}
Wat is het antwoord op 4+4?
\pause

HET IS NIET 9
\pause

MAAR HET IS NATUURLIJK 8
}

\frame{{Voorbeeld van kolommen en boxes}
    \begin{columns}[c] 
		\column{.3\textwidth} 
		\begin{block}{This is a Block}
      MAGISCHE KLEURTJES WAT
    \end{block}
    \column{.3\textwidth}
		\begin{alertblock}{This is a alertBlock}
		EN DIT STAAT HIER SICK!
    \end{alertblock}
		\column{.3\textwidth}
		\begin{exampleblock}{This is a exampleblock}
		BRUUUUUUUUUUUUUUTTT
    \end{exampleblock}
		\end{columns}
}


\end{document}